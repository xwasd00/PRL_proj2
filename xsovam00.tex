\documentclass[12pt,a4paper]{article}
\usepackage{times}
\usepackage[utf8]{inputenc}
\usepackage[czech]{babel}
\usepackage[left=1.5cm, top=1.5cm, text={18cm, 26cm}]{geometry}
\usepackage{graphicx} % figures (\includegraphics...)
\usepackage{float} % figures positioning ([H])
\usepackage[unicode]{hyperref} % \ref{...}
\urlstyle{same} % \ref{...}
\renewcommand{\figurename}{Obrázek}

\usepackage{amsfonts} % natural numbers -> \mathbb{N}

\usepackage{fancyhdr}
\pagestyle{fancy}
\fancyhf{}
\renewcommand{\headrulewidth}{0pt}
\lfoot{PRL Projekt}
\rfoot{Michal Sova}
\cfoot{\thepage}

%\title{Projekt do předmětu PRL - Odd-Even Merge Sort}
%\author{Michal Sova (xsovam00@stud.fit.vutbr.cz)}

\begin{document}
%\maketitle

\section{Přiřazení pořadí preorder vrcholům}
\label{sec:preorder}
Pro algoritmus přiřazení pořadí preorder vrcholům je potřeba vykonat následující kroky:
\begin{itemize}
     \item Vytvoření Eulerovy cesty
     \item Vytvoření pole vah
     \item Výpočet sumy sufixů
     \item Korekce sumy sufixů
 \end{itemize}

Pro funkčnost agoritmu je potřeba spočítat Eulerovu cestu, kterou lze reprezentovat pomocí funkce následníka, t.j. každá hrana si pamatuje svou následující hranu. 
Eulerova cesta je metoda reprezentace stromu pomocí orientovaného grafu kde každá hrana obsahuje dvě orientované hrany stromu. 
Příklad eulerovy cesty je vidět na Obr. \ref{fig:ett}. 
Následně je potřeba pole vah. 
To lze vypočíst tak, že ke každé dopředé hraně dáme váhu 1 a ostatním váhu 0. 
Suma sufixů lze vypočíst pomocí pole vah a pole Etour, reprezentující Eulerovu cestu.
Pole Etour slouží k určení následníka (následujícího prvku) a výstupem je nové pole vah.
Tyto váhy jsou pak upraveny tak, aby určovaly pořadí pro jednotlivé hrany. 
Koncové body těchto seřazených hran a kořenový bod (který je označen zvlášť, jelikož není na výstupu) pak odpovídá výstupu pořadí preorder.

\begin{figure}[H]
    \centering
    \includegraphics[scale=.8]{img/ETT.png}
    \caption{Příklad Eulerovy cesty}
    \label{fig:ett}
\end{figure}
\subsection*{Celková cena}
\label{sub:celkova_cena}
Prostorová složitost se dá vyjádřit jako počet hran v orientvaném grafu binárního stromu a tedy:
\begin{equation}
    p(n) = 2n - 2 \rightarrow \mathcal{O}(n)
\end{equation}
Kde $n$ udává počet vrcholů binárního stromu a $p(n)$ udává potřebný počet procesorů. 
Pro vytvoření eulerovy cesty se dají použít různé algoritmy. Pro tento projekt je použit cyklus pro vytvoření všech následníků dané hrany a tedy:
\begin{equation}
    t(n) = 2n- 2 \rightarrow \mathcal{O}(n)
\end{equation}
Kde $n$ udává počet vrcholů a $t(n)$ udává čas potřebný k provedení daného kroku.
Vytvoření vah je jen jeden krok a tedy:
\begin{equation}
    t(n) = c \rightarrow \mathcal{O}(c)
\end{equation}
Kde $c$ je konstanta.
Sumu sufixů musíme vypočítat pro každou hranu a tedy:
\begin{equation}
    t(n) = log(2n - 2) \rightarrow \mathcal{O}(log(n))
\end{equation}
Následná korekce sumy sufixů je provedena v konstantním čase:
\begin{equation}
    t(n) = c \rightarrow \mathcal{O}(c)
\end{equation}
Výsledná časová složitost odpovídá:
\begin{equation}
    \mathcal{O}(n) + \mathcal{O}(c) + \mathcal{O}(log(n)) + \mathcal{O}(c) = \mathcal{O}(n)
\end{equation}
Výsledná cena $c(n)$ je tedy:
\begin{equation}
    c(n) = p(n) * t(n) = \mathcal{O}(n^2)
\end{equation}

\section{Implementace}
\label{sec:implementace}
Algoritmus je implementován pomocí knihovny \texttt{OpenMPI}. 
Samotné hrany jsou reprezentovány jednotlivými procesy. 
Pořadí procesů (\texttt{rank}) udává pozici hrany v grafu. 
Procesy si vytvoří třídu \texttt{Edge}, která uchovává informace o hraně (číslo hrany, orientaci hrany, počáteční a koncový bod). 
Jednotlivé hrany si následně vypočítají svého následníka pomocí funkce \texttt{create\_etour}. 
Pomocí funkce \texttt{MPI\_Allgather} si procesy zjistí informace o ostatních hranách a každá hrana (proces) si najde svého následníka.

Funkce \texttt{suffix\_sum} pak pomocí \texttt{MPI\_Allgather} simuluje sdílené pole vah a pole Etour (reprezentující Eulerovu cestu) a spočte paralelně pro každý proces odpovídající sumu sufixů pole vah.

Každý proces si zapamatuje svou výslednou váhu a pomocí funkce \texttt{print\_preorder} si dopředné hrany získají hodnotu pořadí preorder daného uzlu. 
Pořadí a odpovídající vrcholy pak posbírá první proces (pomocí \texttt{MPI\_Gather}), který vytiskne na výstup nejdříve kořenový uzel a poté zbytek stromu v pořadí preorder.

\section{Závěr}
\label{sec:závěr}
Nejnáročnější operací, co se týče časové složitosti, je vytváření eulerovy cesty, což určuje výslednou časovou složitost. 
Algoritmus vypisuje pořadí dle očekávání.

\end{document}
